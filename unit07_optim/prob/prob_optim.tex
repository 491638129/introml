\documentclass[11pt]{article}

\usepackage{fullpage}
\usepackage{amsmath, amssymb, bm, cite, epsfig, psfrag}
\usepackage{graphicx}
\usepackage{float}
\usepackage{amsthm}
\usepackage{amsfonts}
\usepackage{listings}
\usepackage{cite}
\usepackage{hyperref}
\usepackage{tikz}
\usepackage{enumerate}
\usepackage{listings}
\usepackage{mathtools}
\lstloadlanguages{Python}
\usetikzlibrary{shapes,arrows}
%\usetikzlibrary{dsp,chains}

\DeclareFixedFont{\ttb}{T1}{txtt}{bx}{n}{9} % for bold
\DeclareFixedFont{\ttm}{T1}{txtt}{m}{n}{9}  % for normal
% Defining colors
\usepackage{color}
\definecolor{deepblue}{rgb}{0,0,0.5}
\definecolor{deepred}{rgb}{0.6,0,0}
\definecolor{deepgreen}{rgb}{0,0.5,0}
\definecolor{backcolour}{rgb}{0.95,0.95,0.92}

%\restylefloat{figure}
%\theoremstyle{plain}      \newtheorem{theorem}{Theorem}
%\theoremstyle{definition} \newtheorem{definition}{Definition}

\def\del{\partial}
\def\ds{\displaystyle}
\def\ts{\textstyle}
\def\beq{\begin{equation}}
\def\eeq{\end{equation}}
\def\beqa{\begin{eqnarray}}
\def\eeqa{\end{eqnarray}}
\def\beqan{\begin{eqnarray*}}
\def\eeqan{\end{eqnarray*}}
\def\nn{\nonumber}
\def\binomial{\mathop{\mathrm{binomial}}}
\def\half{{\ts\frac{1}{2}}}
\def\Half{{\frac{1}{2}}}
\def\N{{\mathbb{N}}}
\def\Z{{\mathbb{Z}}}
\def\Q{{\mathbb{Q}}}
\def\R{{\mathbb{R}}}
\def\C{{\mathbb{C}}}
\def\argmin{\mathop{\mathrm{arg\,min}}}
\def\argmax{\mathop{\mathrm{arg\,max}}}
%\def\span{\mathop{\mathrm{span}}}
\def\diag{\mathop{\mathrm{diag}}}
\def\x{\times}
\def\limn{\lim_{n \rightarrow \infty}}
\def\liminfn{\liminf_{n \rightarrow \infty}}
\def\limsupn{\limsup_{n \rightarrow \infty}}
\def\GV{Guo and Verd{\'u}}
\def\MID{\,|\,}
\def\MIDD{\,;\,}

\newtheorem{proposition}{Proposition}
\newtheorem{definition}{Definition}
\newtheorem{theorem}{Theorem}
\newtheorem{lemma}{Lemma}
\newtheorem{corollary}{Corollary}
\newtheorem{assumption}{Assumption}
\newtheorem{claim}{Claim}
\def\qed{\mbox{} \hfill $\Box$}
\setlength{\unitlength}{1mm}

\def\bhat{\widehat{b}}
\def\ehat{\widehat{e}}
\def\phat{\widehat{p}}
\def\qhat{\widehat{q}}
\def\rhat{\widehat{r}}
\def\shat{\widehat{s}}
\def\uhat{\widehat{u}}
\def\ubar{\overline{u}}
\def\vhat{\widehat{v}}
\def\xhat{\widehat{x}}
\def\xbar{\overline{x}}
\def\zhat{\widehat{z}}
\def\zbar{\overline{z}}
\def\la{\leftarrow}
\def\ra{\rightarrow}
\def\MSE{\mbox{\small \sffamily MSE}}
\def\SNR{\mbox{\small \sffamily SNR}}
\def\SINR{\mbox{\small \sffamily SINR}}
\def\arr{\rightarrow}
\def\Exp{\mathbb{E}}
\def\var{\mbox{var}}
\def\Tr{\mbox{Tr}}
\def\tm1{t\! - \! 1}
\def\tp1{t\! + \! 1}
\def\Tm1{T\! - \! 1}
\def\Tp1{T\! + \! 1}


\def\Xset{{\cal X}}

\newcommand{\one}{\mathbf{1}}
\newcommand{\abf}{\mathbf{a}}
\newcommand{\bbf}{\mathbf{b}}
\newcommand{\dbf}{\mathbf{d}}
\newcommand{\ebf}{\mathbf{e}}
\newcommand{\gbf}{\mathbf{g}}
\newcommand{\hbf}{\mathbf{h}}
\newcommand{\pbf}{\mathbf{p}}
\newcommand{\pbfhat}{\widehat{\mathbf{p}}}
\newcommand{\qbf}{\mathbf{q}}
\newcommand{\qbfhat}{\widehat{\mathbf{q}}}
\newcommand{\rbf}{\mathbf{r}}
\newcommand{\rbfhat}{\widehat{\mathbf{r}}}
\newcommand{\sbf}{\mathbf{s}}
\newcommand{\sbfhat}{\widehat{\mathbf{s}}}
\newcommand{\ubf}{\mathbf{u}}
\newcommand{\ubfhat}{\widehat{\mathbf{u}}}
\newcommand{\utildebf}{\tilde{\mathbf{u}}}
\newcommand{\vbf}{\mathbf{v}}
\newcommand{\vbfhat}{\widehat{\mathbf{v}}}
\newcommand{\wbf}{\mathbf{w}}
\newcommand{\wbfhat}{\widehat{\mathbf{w}}}
\newcommand{\xbf}{\mathbf{x}}
\newcommand{\xbfhat}{\widehat{\mathbf{x}}}
\newcommand{\xbfbar}{\overline{\mathbf{x}}}
\newcommand{\ybf}{\mathbf{y}}
\newcommand{\zbf}{\mathbf{z}}
\newcommand{\zbfbar}{\overline{\mathbf{z}}}
\newcommand{\zbfhat}{\widehat{\mathbf{z}}}
\newcommand{\Ahat}{\widehat{A}}
\newcommand{\Abf}{\mathbf{A}}
\newcommand{\Bbf}{\mathbf{B}}
\newcommand{\Cbf}{\mathbf{C}}
\newcommand{\Bbfhat}{\widehat{\mathbf{B}}}
\newcommand{\Dbf}{\mathbf{D}}
\newcommand{\Gbf}{\mathbf{G}}
\newcommand{\Hbf}{\mathbf{H}}
\newcommand{\Ibf}{\mathbf{I}}
\newcommand{\Kbf}{\mathbf{K}}
\newcommand{\Pbf}{\mathbf{P}}
\newcommand{\Phat}{\widehat{P}}
\newcommand{\Qbf}{\mathbf{Q}}
\newcommand{\Rbf}{\mathbf{R}}
\newcommand{\Rhat}{\widehat{R}}
\newcommand{\Sbf}{\mathbf{S}}
\newcommand{\Ubf}{\mathbf{U}}
\newcommand{\Vbf}{\mathbf{V}}
\newcommand{\Wbf}{\mathbf{W}}
\newcommand{\Xhat}{\widehat{X}}
\newcommand{\Xbf}{\mathbf{X}}
\newcommand{\Ybf}{\mathbf{Y}}
\newcommand{\Zbf}{\mathbf{Z}}
\newcommand{\Zhat}{\widehat{Z}}
\newcommand{\Zbfhat}{\widehat{\mathbf{Z}}}
\def\alphabf{{\boldsymbol \alpha}}
\def\betabf{{\boldsymbol \beta}}
\def\betabfhat{{\widehat{\bm{\beta}}}}
\def\epsilonbf{{\boldsymbol \epsilon}}
\def\mubf{{\boldsymbol \mu}}
\def\lambdabf{{\boldsymbol \lambda}}
\def\etabf{{\boldsymbol \eta}}
\def\xibf{{\boldsymbol \xi}}
\def\taubf{{\boldsymbol \tau}}
\def\sigmahat{{\widehat{\sigma}}}
\def\thetabf{{\bm{\theta}}}
\def\thetabfhat{{\widehat{\bm{\theta}}}}
\def\thetahat{{\widehat{\theta}}}
\def\mubar{\overline{\mu}}
\def\muavg{\mu}
\def\sigbf{\bm{\sigma}}
\def\etal{\emph{et al.}}
\def\Ggothic{\mathfrak{G}}
\def\Pset{{\mathcal P}}
\newcommand{\bigCond}[2]{\bigl({#1} \!\bigm\vert\! {#2} \bigr)}
\newcommand{\BigCond}[2]{\Bigl({#1} \!\Bigm\vert\! {#2} \Bigr)}
\newcommand{\tran}{^{\text{\sf T}}}
\newcommand{\herm}{^{\text{\sf H}}}
\newcommand{\bkt}[1]{{\langle #1 \rangle}}
\def\Norm{{\mathcal N}}
\newcommand{\vmult}{.}
\newcommand{\vdiv}{./}


% Python style for highlighting
\newcommand\pythonstyle{\lstset{
language=Python,
backgroundcolor=\color{backcolour},
commentstyle=\color{deepgreen},
basicstyle=\ttm,
otherkeywords={self},             % Add keywords here
keywordstyle=\ttb\color{deepblue},
emph={MyClass,__init__},          % Custom highlighting
emphstyle=\ttb\color{deepred},    % Custom highlighting style
stringstyle=\color{deepgreen},
%frame=tb,                         % Any extra options here
showstringspaces=false            %
}}

% Python environment
\lstnewenvironment{python}[1][]
{
\pythonstyle
\lstset{#1}
}
{}

% Python for external files
\newcommand\pythonexternal[2][]{{
\pythonstyle
\lstinputlisting[#1]{#2}}}

% Python for inline
\newcommand\pycode[1]{{\pythonstyle\lstinline!#1!}}

\begin{document}

\title{Introduction to Machine Learning\\
Homework 5:  Gradient Calculations and Nonlinear Optimization}
\author{Prof. Sundeep Rangan}
\date{}

\maketitle

%Submit answers only to problems 1, 2, 3 and 4(b) and (c).  You do not need to answer  4(a).
%But, make sure you know how to do all the problems.
%I looked at the solution, I think the students should be able to do all of them. 

\begin{enumerate}

\item Suppose we want to fit a model,
\[
    \hat{y} = \frac{1}{w_0 + \sum_{j=1}^d w_jx_j},
\]
for parameters $\wbf$. Given training data $(\xbf_i,y_i)$, $i=1,\ldots,n$,
a nonlinear least squares fit could use the loss function,
\[
    J(\wbf) = \sum_{i=1}^n \left[y_i -
        \frac{1}{w_0 + \sum_{j=1}^d w_jx_{ij}}\right]^2
\]
\begin{enumerate}[(a)]
\item Find a function $g(\zbf)$ and matrix $\Abf$ such that
the loss function is given by,
\[
    J(\wbf) = g(\zbf), \quad \zbf = \Abf\wbf,
\]
and $g(\zbf)$ is factorizable, meaning $g(\zbf) = \sum_i g_i(z_i)$ for
some functions $g_i(z_i)$.

\item What is the gradient $\nabla J(\wbf)$?

\item What is the gradient descent update for $\wbf$?

\item Write a few lines of python code to compute the loss function $J(\wbf)$ and
$\nabla J(\wbf)$.
\end{enumerate}

\item In this problem, we will see why gradient descent can often exhibit
very slow convergence, even on apparently simple functions.
Consider the objective function,
\[
    J(\wbf) = \frac{1}{2}b_1w_1^2 + \frac{1}{2}b_2w_2^2,
\]
defined on a vector $\wbf=(w_1,w_2)$ with constants $b_2 > b_1 > 0$.
\begin{enumerate}[(a)]
  \item What is the gradient $\nabla J(\wbf)$?
  \item What is the minimum $\wbf^* = \argmin_{\wbf} J(\wbf)$?
  \item Part (b) shows that we can minimize $J(\wbf)$ easily by hand.
  But, suppose we tried to minimize it via gradient descent.
  Show that the gradient descent update of $\wbf$ with a step-size $\alpha$
has the form,
\[
    w_1^{k+1} = \rho_1 w_1^k, \quad w_2^{k+1} = \rho_2 w_2^k,
\]
for some constants $\rho_i$, $i=1,2$.  Write $\rho_i$ in terms of
$b_i$ and the step-size $\alpha$.
  \item For what values $\alpha$ will gradient descent converge to the minimum? That is, what step sizes guarantee that $\wbf^k \arr \wbf^*$.

\item Take $\alpha = 2/(b_1+b_2)$.  It can be shown that this choice of $\alpha$ results in the fastest convergence.  You do not need to show this.
But, show that with this selection of $\alpha$,
\[
    \|\wbf^k\| = C^k \|\wbf^0\|, \quad C = \frac{\kappa-1}{\kappa+1 }, \quad
    \kappa = \frac{b_2}{b_1}.
\]
The term $\kappa$ is called the \emph{condition number}.  The above calculation
shows that when $\kappa$ is very large, $C \approx 1$ and the convergence
of gradient descent is slow.  In general, gradient descent performs poorly
when the problems are ill-conditioned like this.
\end{enumerate}

\item \emph{Matrix minimization}.  Consider the problem of finding a matrix
$\Pbf \in \R^{m \x m}$ to minimize the loss function,
\[
    J(\Pbf) = \sum_{i=1}^n \left[ \frac{z_i}{y_i} - \ln(z_i) \right],
    \quad
    z_i = \xbf_i\tran \Pbf \xbf_i.
\]
The problem arises in wireless communications where an $m$-antenna receiver wishes
to estimate a spatial covariance matrix $\Pbf$ from $n$ power measurements.
In this setting, $y_i > 0$ is the $i$-th receive power measurement and $\xbf_i$
is the beamforming direction for that measurement.  In reality, the quantities
would be complex, but for simplicity we will just look at the real-valued case.
See the following article for more details:
\begin{quote}
  Eliasi, Parisa A., Sundeep Rangan, and Theodore S. Rappaport. ``Low-rank spatial channel estimation for millimeter wave cellular systems," \emph{IEEE Transactions on Wireless Communications} 16.5 (2017): 2748-2759.
\end{quote}
\begin{enumerate}[(a)]
\item What is the gradient $\nabla_{\Pbf} z_i$?

\item What is the gradient $\nabla_{\Pbf} J(\Pbf)$?

\item Write a few lines of python code to evaluate $J(\Pbf)$ and
$\nabla_{\Pbf} J(\Pbf)$ given data $\xbf_i$ and $y_i$.  You can use a for loop.

\item See if you can rewrite (c) without a for loop.  You will need Python broadcasting.
\end{enumerate}

\item \emph{Nested optimization}.  Suppose we are given a loss function
$J(\wbf_1,\wbf_2)$ with two parameter vectors $\wbf_1$ and $\wbf_2$.
In some cases, it is easy to minimize over one of the sets of parameters, say
$\wbf_2$, while holding the other parameter vector (say, $\wbf_1$) constant.
In this case, one could perform the following \emph{nested} minimization:
Define
\[
    J_1(\wbf_1) := \min_{\wbf_2} J(\wbf_1,\wbf_2), \quad
    \wbfhat_2(\wbf_1) := \argmin_{\wbf_2} J(\wbf_1,\wbf_2),
\]
which represent the minimum and argument of the loss function over $\wbf_2$
holding $\wbf_1$ constant.   Then,
\[
    \wbfhat_1 = \argmin_{\wbf_1} J_1(\wbf_1) = \argmin_{\wbf_1} \min_{\wbf_2}
    J(\wbf_1,\wbf_2).
\]
Hence, we can find the optimal $\wbf_1$ by minimizing $J_1(\wbf_1)$
instead of minimizing $J(\wbf_1,\wbf_2)$ over $\wbf_1$ and $\wbf_2$.
\begin{enumerate}[(a)]
\item Show that the gradient of $J_1(\wbf_1)$ is given by
\[
    \nabla_{\wbf_1} J_1(\wbf_1) = \left. \nabla_{\wbf_1} J(\wbf_1,\wbf_2)\right|_{\wbf_2=\wbfhat_2}.
\]
Thus, given $\wbf_1$, we can evaluate the gradient from (i) solve the minimization
$\wbfhat_2:= \argmin_{\wbf_2} J(\wbf_1,\wbf_2)$; and (ii) take the gradient
$\nabla_{\wbf_1} J(\wbf_1,\wbf_2)$ and evaluate at $\wbf_2 = \wbfhat_2$.

\item Suppose we want to minimize a nonlinear least squares,
\[
    J(\abf,\bbf) := \sum_{i=1}^n \left( y_i -
        \sum_{j=1}^d b_j e^{-a_jx_i} \right)^2,
\]
over two parameters $\abf$ and $\bbf$.  Given parameters $\abf$,
describe how we can minimize over $\bbf$.  That is, how can we compute,
\[
    \hat{\bbf} := \argmin_{\bbf} J(\abf,\bbf).
\]

\item In the above example, how would we compute the gradients,
\[
    \nabla_\abf J(\abf,\bbf).
\]

\end{enumerate}

\end{enumerate}
\end{document}
